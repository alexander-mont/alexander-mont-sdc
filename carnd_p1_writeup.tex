\documentclass{article}
\usepackage{graphicx}
\usepackage{tipa}
\date{February 24, 2018}

\begin{document}
\title{Finding Lane Lines, Part 1}
\author{Alexander Mont}
\maketitle
\section{Pipeline Overview}

Here is a simple project for finding the lane lines on a road image. The pipeline consisted of the following steps:

\begin{enumerate}
\item First, we convert the image to grayscale. This is because the subsequent algorithms to be run on this, such as Canny edge detection, only work on grayscale images. Note that we do this by using the OpenCV function \texttt{cv2.cvtColor(img, cv2.COLOR\_RGB2GRAY}, which treats all of the three color channels (red, green, blue) equally.
\item Then, we apply Canny edge detection. Canny edge detection works by computing the \emph{gradient} (difference between adjacent pixels) of the image, and highlights only the places where this gradient is highest - i.e., where there is an edge. Note that this effectively turns a segment of a lane line (a dotted line) into a quadrilateral, and turns a solid lane line into two edges.
\item The output of Canny edge detection is just a pixel bitmap showing which pixels are part of edges. Now we need to turn these pixels into lines. To do this we use the Hough transform, and algorithm which does the following:
\begin{enumerate}
\item Considers the ``space of possible lines'' defined by $\theta$ (the angle of the line), and $\rho$ (the shortest distance of the line from the origin) - this is referred to as the ``Hough space''.
\item Each highlighted point defines a curve in the Hough space representing the possible lines that contain that point.
\item Points in Hough space where lots of the above lines intersect represent lines in the original space (the more Hough curves intersect at that point in Hough space, the longer that line is)
\end{enumerate}
\item This process is likely to generate multiple lines for each lane line, so we must now combine these lines into just two lines - one on the left side and one on the right side. We do this as follows:
\begin{enumerate}
\item Divide up the lines into two categories: one with negative slope (on the left) and one with positive slope (on the right). The negative slope is on the left because the X axis goes from left to right, but the Y axis goes from the top dow.
\item In each of these categories, compute the \emph{smallest} (absolute value of) slope, where slope here is defined as $\frac{x}{y}$ - note that this is different from how slope is usually defined. We do it this way so that a nearly vertical line will not cause numerical issues due to the extremely high slope (and we expect nearly vertical lines to be more frequent than nearly horizontal lines(
\item On each side, we consider only Hough lines with a slope of no more than 0.2 away from this ``smallest slope''. The purpose of this is to filter out any lane lines other than the lines that bound the lane the car is in (a lane line multiple lanes over will have a higher slope)
\item We compute the average of the slopes of these lines to get the overall slope of the lane line, and then compute the intercept by fitting a line of the given slope through the endpoints of all the Hough lines.
\end{enumerate}
\item Then we draw the overall lane lines.
\end{enumerate}

It is also possible to use this algorithm to annotate the lane lines in a video by applying the above algorithm to each frame in the video. When I did this I noticed that the drawn lane lines jumped around a lot from frame to frame even though the actual lane lines do not appear to move that much. Thus I changed my algorithm in this case to compute the slope of each line in a given frame as (0.95 * slope of line from previous frame) + (0.05 * observed average slope of the Hough lines as described above). Thus this significantly dampens the frame-to-frame vibrations.

\section{Discussions and Improvements}

Here are some of the shortcomings of this algorithm and ways it could be improved. I plan to implement these and other improvements in a future project.

\begin{enumerate}
\item The slopes of the Hough lines appear to change a lot from frame to frame spuriously. It is likely that this is because the edges obtained by the Canny edge detection are only one pixel wide (all the pixels in the "middle" of the lane line are thrown out) so a change in even a few of these pixels may significantly change the slope. A better approach may be to note that we don't really care about the \emph{edges} of the lane line per se (the ``lane line'' is really a thin quadrilateral) what we care about is the overall lane line. Therefore, a better approach may be to not use Canny edge detection at all. A better approach may be to do the following:
\begin{enumerate}
\item Use thresholding to identify which pixels in the image are likely part of lane lines. for instance, any white or yellow pixels are likely part of lane lines. So the thresholding could be based on just the red and green color channels, since yellow has high red and green but low blue.
\item Use a flood fill or similar algorithm to segment these ``likely pixels'' into contiguous groups.
\item Try to identify which contiguous groups are lane lines on the road (as opposed to other objects such as white or yellow cars). Note that a lane line or lane line segment is expected to be long and thin. Thus, if one took all the $(x,y)$ coordinates of pixels in a lane line and did PCA on them, one would expect to see one very large principal component and one much smaller principal component. This pattern could be easily identified.
\item Once we have a contiguous group of pixels that represents a lane line, we can fit a line through it (e.g. using a least squares fit) to get the slope and intercept.
\end{enumerate}
\item The smoothing technique used between frames in the video works well for the test videos in this project where the car is moving roughly straight, but may not work well in a scenario where the car is changing lanes so the slope of the lane lines actually is changing significantly - it might be slow to catch up. A better solution here may be a Kalman filter which stores as its state both the current slope and a rate of change in slope. Thus if the car was e.g. changing lanes, where the lane line's slope is smoothly changing, the filter would pick up on this and track it.
\item The identifying of each lane line by a slope and intercept assumes that it is a straight line. This may not be true if the road is curving or if there is a hill in front of the car This is a major flaw because both of these situations are likely to require action by a self-driving car, so the car would want to know about it. A solution here may be to, after the lane line segments on the road have been identified as described above, to use some sort of clustering algorithm to find "clusters" of these line segments in Hough space, and assume that similar clusters represent the same lane lines. This would also mean that the algorithm wouldn't need to assume that there are two primary lane lines, one on each side - it could detect other things that look like lane lines.
\end{enumerate}

\end{document}